\documentclass[a4paper,12pt]{article}
\usepackage[ngerman]{babel} 

\usepackage[T1]{fontenc}
\usepackage[utf8]{inputenc}

\begin{document}


\title{Neuronale Netzwerke}
\section{Was sind Neuronale Netze}
Ein neuronales Netz (NN) oder auch ein künstliches neuronales Netz (KNN), ist ein informationsverarbeitendes System. Es besteht aus einer Vielzahl einfacher Einheiten (Neuronen, Units), die sich Informationen in Form der Aktivierung der Zellen über gerichtete Verbindungen (connections, links) zusenden.
\subsection{Vorteile}
\begin{center}
\begin{enumerate}
\item Können bessere Ergebnisse liefern als existierende statistische Ansätze
\item Für große Datenmengen und viele Datendimensionen können sinnvolle Ergebnisse ermittelt werden
\end{enumerate}
\end{center}
\subsection{Nachteile}
\begin{center}
\begin{enumerate}
\item Es bedarft vieler Trainingsdaten um ein allgemeingültiges gutes Ergebnis zu berechnen
\item Lange Trainingszeiten
\end{enumerate}
\end{center}
\end{document}