\documentclass[main.tex]{subfiles}
\begin{document}
\author{Philipp Nickel}

\noindent 
%Eine kurze (!) Zusammenfassung des Berichtes!
In diesem Projekt geht es um die Entwicklung eines Intrusion Detection Systems(IDS). Das genaue Ziel war es anhand eines ausgewählten Datensatz, in diesem Fall dem CTU-13 Datensatz ein IDS zu entwickeln. Die Intrusion Detection  wird durch einen gewählten Machine Learning Algorithmus durchgeführt. In diesem Projekt wurden dafür die Algorithmen des K-Nearest Neighbour Verfahren genutzt. Dabei werden die lineare Suche und das Balltree Verfahren genutzt. Um auf das bestmögliche Ergebnis zu kommen wurden die Werte für K und andere Parameter variiert. Das beste Ergebnis lässt sich anhand der Metriken Recall und Specifity bestimmen. Die besten Ergebnisse lassen sich daraufhin anhand der resultierenden True Positve Rate, False Positive Rate, True Negativ Rate und False Negative Rate bestimmen.    

\end{document}