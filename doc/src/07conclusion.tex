\documentclass[main.tex]{subfiles}
\author{Philipp Nickel}
\begin{document}
\section{Fazit und Ausblick}



Die Ergebnisse tendieren dazu, sich mit zunehmendem k zu verbessern, wie bereits im Kapitel 6 präsentiert wurde.
Jedoch sind auch die damit erzielten Ergebnisse nicht optimal, da auch eine True Positive Rate von 99,99\% nicht ausreichend ist. Schließlich ist  bei entsprechend vielen Alarmen die Anzahl an Fehlalarmen oder nicht erkannten Angriffen immer noch zu hoch. Es bleibt zu beobachten, wie sich die Genauigkeit und True Positive Rate mit immer weiter zunehmendem k verändern.  
\\
Im Laufe des Projektes mussten wir feststellen, dass sich der CTU-13 Datensatz nur bedingt für Intrusion Detection eignet. Zwar erfüllt der Datensatz alle Voraussetzungen, die für Intrusion Detection nötig sind, jedoch traten auch einige Probleme auf. Unter anderem konnten nicht alle Szenarien im IDS genutzt werden. Allein die Größe einiger Szenarien führte zu \textit{OutOfMemory} Fehlern. So wurde in diesem Projekt komplett auf das Szenario 10 des CTU-13 Datensatzes verzichtet. Grund dafür war die Größe von 70 GB, bei der eine Verarbeitung und Analyse die vorhandenen Ressourcen und Kapazitäten überschritten hätte.
\\
Demzufolge bleibt die Aufgabe für zukünftige Projekte mit diesem Datensatz, die genannten Probleme zu lösen und somit eine komplette Analyse des gesamten Datensatzes durchzuführen.
\end{document}

