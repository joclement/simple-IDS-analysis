\documentclass[main.tex]{subfiles}
\begin{document}
\section{Einleitung}
%Das hier ist die Einleitung!
Ein automatisiertes Auto wird ohne höchste Sicherheitsgarantien niemals allein fahren können. Die Versorgung der Bevölkerung mit wesentlichen Gütern kann industriellen Steuerungssystemen ohne funktionierende Absicherung nicht überantwortet werden. Die Wertschöpfung der deutschen Wirtschaft durch Vorsprung in Know-how und Technik darf nicht durch Industriespionage gefährdet werden..... \\
%\cite{•}
Durch Statements wie diesen wird deutlich, wie weit fortgeschritten die Digitaliserung bereits ist und wie weit sie noch reichen wird. Jedoch wird es auch unmittelbar deutlich, dass die Komplexität der Bedrohungslage, sowie die damit einhergehenden Gefahren zunehmen. Somit stellt sich die Frage der Sicherheit nicht mehr nur nebenbei. Vielmehr ist die Informationssicherheit eine wesentliche Vorbedingung für das Gelingen der Digitalisierung geworden.
\\ 
Eine Möglichkeit um die Informationssicherheit zu verbessern ist die Nutzung eines Intrusion Detection System. Diese dienen primär der Erkennung von Angriffen und zeichnen diese als Log-Dateien auf.
\\
Demzufolge behandelte dieses Projekt die Entwicklung eines IDS auf Basis eines bestimmten Datensatzes. Im folgenden Bericht wird die Planung des Projektablaufs, der gewählte Datensatz selbst und das daraufbasierend entwickelte IDS, sowie die daraus resultierenden Ergebnisse und Schlussfolgerungen.
\\

\end{document}

% Kurz beschreiben, warum dies ein wichtiges Thema ist, das Beachtung fordert.
% Am besten mit ein paar Statistiken belegen (So und so viele IT-Sicherheitsprobleme gab es allein im Jahr x...)
% Kurz die Gliederung vorstellen

